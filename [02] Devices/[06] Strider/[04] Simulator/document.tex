\documentclass{article}
\usepackage[utf8]{inputenc}
\usepackage{amsmath}
\usepackage{amsfonts}
\usepackage{amssymb}
\usepackage{graphicx}
\usepackage{geometry}
\geometry{margin=1in}

\title{Strider Robot Simulator Documentation}
\author{Your Name}
\date{\today}

\begin{document}

\maketitle

\section{Introduction}
This document describes the Strider robot simulator project.

\section{System Overview}
The Strider simulator is designed to model the behavior of a walking robot.

\subsection{Key Features}
\begin{itemize}
    \item Physics simulation
    \item Real-time visualization
    \item Configurable parameters
\end{itemize}

\section{Mathematical Model}
The robot's movement can be described by the following equations:

\begin{equation}
    F = ma
\end{equation}

Where:
\begin{align}
    F &= \text{Force applied} \\
    m &= \text{Mass of the robot} \\
    a &= \text{Acceleration}
\end{align}

\section{Implementation}
The simulator is implemented in Python using the following structure:

\begin{verbatim}
class StriderSimulator:
    def __init__(self):
        # Initialize simulation parameters
        pass
    
    def step(self):
        # Run one simulation step
        pass
\end{verbatim}

\section{Results}
Results and analysis will be documented here.

\section{Conclusion}
The Strider simulator provides a comprehensive platform for robot development.

\end{document}
